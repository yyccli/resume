% !TEX TS-program = xelatex
% !TEX encoding = UTF-8 Unicode
% !Mode:: "TeX:UTF-8"

\documentclass{resume}
\usepackage{zh_CN-Adobefonts_external} % Simplified Chinese Support using external fonts (./fonts/zh_CN-Adobe/)
% \usepackage{NotoSansSC_external}
% \usepackage{NotoSerifCJKsc_external}
% \usepackage{zh_CN-Adobefonts_internal} % Simplified Chinese Support using system fonts
\usepackage{linespacing_fix} % disable extra space before next section
\usepackage{cite}

\begin{document}
\pagenumbering{gobble} % suppress displaying page number

\name{李泱丞}

\basicInfo{
  \email{ycli98@outlook.com} \textperiodcentered\ 
  \phone{(+86) 18001560090} \textperiodcentered\ 
  \github[yyccli]{https://github.com/yyccli}}
 
\section{教育背景}
\datedsubsection{\textbf{上海交通大学},上海}{2020 -- 至今}
\textit{在读硕士研究生}\ 计算机科学与技术,预计2023年6月毕业
\datedsubsection{\textbf{中南大学},长沙,湖南}{2016 -- 2020}
\textit{学士}\ 计算机科学与技术

\section{研究经历}
Synpose: A Large-Scale and Densely Annotated Synthetic Dataset for Human Pose Estimation in Classroom
Zefang Yu,\textbf{Yangcheng Li},Yicheng Liu,Ting Liu,Yuzhuo Fu. \textbf{ICASSP} 2022
\begin{itemize}
  \item 提出SynPose:一个面向课堂和会议场景的虚拟数据集,并用于人体姿态估计任务
  \item 提出Classroom-style Transfer Generative Adversarial Network(CTGAN)来更好的进行虚拟域和真实域的风格迁移
  \item 搭建一个网站提供数据集相关细节:{\underline{\url{https://yuzefang96.github.io/SynPose/}}}
\end{itemize}
个人贡献:
\begin{itemize}
  \item 完善数据集采集工程代码,实现CTGAN模型并进行消融实验,建设数据集网站部分内容
\end{itemize}

\section{实习/项目经历}
\datedsubsection{\textbf{参与开源社区项目:{\underline{\href{https://github.com/open-mmlab/mmaction2}{mmAction2}}}}}{2022年5月}
\begin{itemize}
  \item 参与相关issue讨论:{\underline{\href{https://github.com/open-mmlab/mmaction2/issues/1624}{\#1624}}},{\underline{\href{https://github.com/open-mmlab/mmaction2/issues/1628}{\#1628}}},{\underline{\href{https://github.com/open-mmlab/mmaction2/issues/1640}{\#1640}}},{\underline{\href{https://github.com/open-mmlab/mmaction2/issues/1662}{\#1662}}}
  \item 提交若干pull request:{\underline{\href{https://github.com/open-mmlab/mmaction2/pull/1622}{\#1622}}},{\underline{\href{https://github.com/open-mmlab/mmaction2/pull/1627}{\#1627}}},{\underline{\href{https://github.com/open-mmlab/mmaction2/pull/1630}{\#1630}}}
\end{itemize}

\datedsubsection{\textbf{研究生阶段研究项目}}{2021年10月 -- 至今}
基于DEtection TRansformer(DETR)的时序视频动作检测算法探究
\begin{itemize}
  \item 基于DETR架构,提出一种新的时序视频动作检测模型框架
  \item 实现端到端的单阶段训练,避免了许多人工设计
  \item 针对视频动作特点对模型进行针对性改进,取得较好的检测结果
\end{itemize}

\datedsubsection{\textbf{本科阶段毕业设计}}{2020年3月 -- 2020年6月}
基于注意力的目标检测及其在尿沉渣有形成分分析中的应用
\begin{itemize}
  \item 将尿沉渣有形成分分析视作目标检测问题,区别于传统的手工方法
  \item 采用Faster RCNN+FPN进行目标检测分析,检测效果良好
\end{itemize}

\datedsubsection{\textbf{中科南京人工智能创新研究院} \textit{暑期实习}}{2019年7月 -- 2019年9月}
功率仪上的数字识别算法
\begin{itemize}
  \item 使用灰度化,阈值处理和形态学滤波等方法对原始图像进行预处理,改善数据质量
  \item 训练CNN模型用于识别单个数字,最终识别效果可接近99\%并能够较好的检测小数点
\end{itemize}

% Reference Test
%\datedsubsection{\textbf{Paper Title\cite{zaharia2012resilient}}}{May. 2015}
%An xxx optimized for xxx\cite{verma2015large}
%\begin{itemize}
%  \item main contribution
%\end{itemize}

\section{IT技能}
% increase linespacing [parsep=0.5ex]
\begin{itemize}
  \item 编程语言:C++ == python > C\#
  \item 平台:Linux,PyTorch
  \item 英语水平:六级(CET\_6)578分
\end{itemize}

\section{获奖情况}
\datedline{中南大学优秀学生}{2016 -- 2017}
\datedline{中南大学校级二等,三等奖学金}{2017,2018}

%% Reference
%\newpage
%\bibliographystyle{IEEETran}
%\bibliography{mycite}
\end{document}
